望ましいフェアリングを設計する上で
我々が注目すべき物理量である熱量はJ、つまり仕事と同じ次元を持つ。高校、大学1年で学ぶ熱力学は大雑把に書けば
温度(K)、圧力(Pa)といった補助単位を用いながら熱量と仕事の変換を記述する学問であろう。それに対して輻射電熱の問題
(これは伝熱工学に属する)は熱量そのものの移動を考えることになる。\par
熱量の移動形態には3種類ある。熱伝導、熱伝達、熱輻射である。前2つに関してはほかのテキストや後で書く(つもり)
資料に譲ることにして、熱輻射について説明する。物体は、常に熱エネルギーを電磁波として放出している。この現象を
熱輻射という。1秒あたりに放出される熱エネルギーは絶対温度の4乗に比例する。比例定数を考慮して立式してみよう。
$\epsilon$を輻射のしやすさを表す0から1の値を取る無次元数とし、$\sigma$をステファン-ボルツマン定数とすると
放出される熱エネルギー(単位はWになる)をQとすると$Q = \epsilon \sigma T^{4}$となる。