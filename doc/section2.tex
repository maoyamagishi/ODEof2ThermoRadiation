\documentclass[C:\wrks\ODEof2ThermoRadiation\doc\ExaminationOfRadiationEff.tex]{subfiles}
\begin{document}
次は2つの物体が相互作用する場合…パイロットとフェアリングをモデル化し、どのように温度変化していくかを見ていこう。
ここでは2物体間の輻射のみ考える。簡単のため、物体から放出される電磁波はすべてもう片方の物体に当たると仮定する。
これは、物体の形状・位置関係としては平板2枚が十分近づいている状態(無限に大きい平板と考えてもよい)に相当する。\par
物体をそれぞれ物体1、物体2と呼ぶことにし、その温度を$T_{1}, T_{2}$とする。するとこの系は連立非線形常微分方程式に
支配される系となり、その方程式は
\begin{align*}
  m_{1}c_{1} \cfrac{dT_{1}}{dt} = \varepsilon_{1} \sigma \lparen T_{2}^{4} - T_{1}^{4} \rparen + q_{in1}(t)\\
  m_{2}c_{2} \cfrac{dT_{2}}{dt} = \varepsilon_{2} \sigma \lparen T_{1}^{4} - T_{2}^{4} \rparen + q_{in2}(t)\\
\end{align*}
となる。この解析解を求めるのは(たぶん)かなり難しい。そのため、ここからは数値計算に頼ってやっていくことになる。\par

まず、2物体の熱容量が等しく、それぞれの初期温度$T_{1} (0) = 360,T_{2} (0) = 300$とし、
外部との熱のやり取りがない($q_{1in} = q_{2in} = 0$)という条件で計算した結果が以下である。

さて、この結果の妥当性について検証してみよう。この系は熱的に閉じた系である。そのため系の熱量の総和は
時間によらない。
\begin{align*}
  Q_{sum}(0) &= m_{1} c_{1} T_{1}(0) + m_{2} c_{2} T_{2}(0)\\
  &=360 + 300 = 660\\
  Q_{sum}(\infty) &= m_{1} c_{1} T_{1}(\infty) + m_{2} c_{2} T_{2}(\infty)\\
  &=330 + 330 = 660\\
\end{align*}
という訳で、大丈夫そうである。またこの結果から、同じような条件下ではt->$\infty$で2つの
物体の温度が等しくなりそうだ。以降これを仮定した上で話を進めていこう。では、
任意の物質の質量、比熱ではどうだろうか?$Q_{sum}$が一定であること、$T_{1}(\infty) = T_{2}(\infty)$
から以下の式を導出できる。
\begin{align*}
  \begin{pmatrix}
    m_{1} c_{1}& m_{2} c_{2}\\
    -1&1
  \end{pmatrix}
  \begin{pmatrix}
    T_{1}(\infty)\\
    T_{0}(\infty)
  \end{pmatrix}
  =
  \begin{pmatrix}
    m_{1} c_{1} T_{1}(0) + m_{2} c_{2} T_{2}(0) \\
    0
  \end{pmatrix}
\end{align*}
ここから直ちに$T_{0}(\infty) = T_{1}(\infty) = \cfrac{m_{1} c_{1} T_{1}(0) + m_{2} c_{2} T_{2}(0)}{m_{1} c_{1} + m_{2} c_{2}}$
ここまでは高校で扱うような「水の中に熱した鉄球を入れる」ような問題とそう変わらない。
(厳密なことを書けば、例に挙げたようなものは熱伝達であるのでニュートンの冷却法則に従う。そのため
温度は指数関数の線形和で表されるような関数となって、曲線の形は変わるだろう。)しかし、
真に輻射の面白い性質はここからである。
\end{document}