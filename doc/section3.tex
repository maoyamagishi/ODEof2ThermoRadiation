\documentclass[C:\wrks\ODEof2ThermoRadiation\doc\ExaminationOfRadiationEff.tex]{subfiles}
\begin{document}
結論から書けば、あるパラメータを導入することで、輻射伝熱によって熱をやりとりしている系は
定常状態でそれぞれの物体が異なる温度を取りうる。つまり、$T_{1}(\infty) \neq T_{2}(\infty)$
が許されるのである!\par
そのパラメータは、形態係数と呼ばれる。これは、物体の形状や他の物体との距離によって算出される値で、
雑に書けば放出される電磁波の届きやすさのパラメータである。記号は$F$で表すのが一般的であり、
定義としては、物体$i$から輻射により放出される単位時間当たりの熱量を$Q_{i-out}$とし、
物体$j$が吸収する輻射のうち物体$i$からでたものを$Q_{ij}$と書いたときその比となる。つまり、
\begin{equation*}
  F_{ij} = \cfrac{Q_{ij}}{Q_{i-out}}
\end{equation*}
である
\end{document}